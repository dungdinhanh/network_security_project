\documentclass[conference]{IEEEtran}
\IEEEoverridecommandlockouts
% The preceding line is only needed to identify funding in the first footnote. If that is unneeded, please comment it out.
%\usepackage{cite}
\usepackage{amsmath,amssymb,amsfonts}
\usepackage{algorithmic}
\usepackage{graphicx}
\usepackage{textcomp}

\usepackage{lipsum}
\usepackage[ruled,linesnumbered,resetcount,noend,noline]{algorithm2e}
\usepackage{multicol}
%\usepackage{algpseudocode}
\usepackage{longtable}


\usepackage{enumitem}

%\usepackage{subfig}
\usepackage{booktabs}
\usepackage[table]{xcolor}
\usepackage{pdflscape}
\usepackage{longtable}
\usepackage{multirow}
%\usepackage{geometry}
\usepackage{threeparttablex}
\usepackage[backend=bibtex,maxnames=1]{biblatex}
\usepackage{subcaption}
%\usepackage[labelformat=parens,labelsep=quad,skip=3pt]{caption}
\usepackage{caption}
\captionsetup[subfigure]{subrefformat=simple,labelformat=simple}
\usepackage{subfiles}
\usepackage{graphicx}
\usepackage{acronym}
%\usepackage[compatibility=false]{caption}
%\usepackage[font=small,labelfont=bf,tableposition=top]{caption}
%\usepackage[caption=false]{subfig}
%\usepackage{subcaption}
%\usepackage[numbers]{natbib}
%\usepackage{hyperref}
\usepackage{breakurl}
\usepackage{url}
\usepackage[bookmarks=true]{hyperref}
\usepackage{enumitem}



\newlist{steps}{enumerate}{1}
\setlist[steps, 1]{label= \textbf{Step \arabic*:}}

% The following packages can be found on http:\\www.ctan.org
%\usepackage{graphics} % for pdf, bitmapped graphics files
%\usepackage{epsfig} % for postscript graphics files
%\usepackage{mathptmx} % assumes new font selection scheme installed
%\usepackage{times} % assumes new font selection scheme installed
%\usepackage{amsmath} % assumes amsmath package installed
%\usepackage{amssymb}  % assumes amsmath package installed
\bibliography{report} 
%\addbibresource{report}

\title{\LARGE \bf
Survey on Wireless Local Area Network security
}

%\author{ \parbox{3 in}{\centering Huibert Kwakernaak*
%         \thanks{*Use the $\backslash$thanks command to put information here}\\
%         Faculty of Electrical Engineering, Mathematics and Computer Science\\
%         University of Twente\\
%         7500 AE Enschede, The Netherlands\\
%         {\tt\small h.kwakernaak@autsubmit.com}}
%         \hspace*{ 0.5 in}
%         \parbox{3 in}{ \centering Pradeep Misra**
%         \thanks{**The footnote marks may be inserted manually}\\
%        Department of Electrical Engineering \\
%         Wright State University\\
%         Dayton, OH 45435, USA\\
%         {\tt\small pmisra@cs.wright.edu}}
%}

%\author{Huibert Kwakernaak$^{1}$ and Pradeep Misra$^{2}$% <-this % stops a space
%\thanks{*This work was not supported by any organization}% <-this % stops a space
%\thanks{$^{1}$H. Kwakernaak is with Faculty of Electrical Engineering, Mathematics and Computer Science,
%        University of Twente, 7500 AE Enschede, The Netherlands
%        {\tt\small h.kwakernaak at papercept.net}}%
%\thanks{$^{2}$P. Misra is with the Department of Electrical Engineering, Wright State University,
%        Dayton, OH 45435, USA
%        {\tt\small p.misra at ieee.org}}%
%}

\author{
	\IEEEauthorblockN{
		Dinh Anh Dung \IEEEauthorrefmark{2},
		Bui Anh Vu \IEEEauthorrefmark{2}
	}
	\IEEEauthorblockA{
		\IEEEauthorrefmark{2} School of Information and Communication Technology, Hanoi University of Science and Technology, Vietnam\\	
		Email: $\left\{\text{dinhanhdung, buianhvu}\right\}$@gmail.com \\	
		SID: 20140774, 20144875	
	}
}

\begin{document}

%\bibliographystyle{ieeetr}

\maketitle
\thispagestyle{empty}
\pagestyle{empty}

\newacro{WEP}{Wired Equivalent Privacy}
\newacro{TKIP}{Temporal Key Integrity Protocol}
\newacro{LAN}{Local Area Network}
\newacro{WLAN}{Wireless Local Area Network}
\newacro{IEEE}{Institue of Electrical and Electronics Engineers}
\newacro{WPA}{Wi-fi Protected Access}
\newacro{CPU}{Central Processing Unit}
\newacro{AES}{Advanced Encryption Standard}
\newacro{RC4}{Rivest Cipher 4}
\newacro{IV}{Initial Vector}
\newacro{Xor}{Exclusive-or}
\newacro{STA}{Station}
\newacro{AP}{Access Point}
\newacro{PCMCIA}{Personal Computer Memory Card International Association}
\newacro{CRC}{Cycle Redundant Check}
\newacro{MAC}{Medium Access Control}
\newacro{EAP}{Extensible Authenticate Protocol}
\newacro{TK}{Temporal Key}
\newacro{P1K}{Phase 1 Key}
\newacro{MIC}{Message Integrity Code}
\newacro{AES}{Advanced Encryption Standard}
\newacro{CBCMAC}{Cipher Block Chaining Message Authentication Code Protocol}
\newacro{CCMP}{Counter Mode Chain Cipher Block Chaining Message Authentication Code Protocol}
\newacro{CTR-AES}{Counter Mode AES}
\newacro{NIST}{National Institute of Standards and Technology}

%%%%%%%%%%%%%%%%%%%%%%%%%%%%%%%%%%%%%%%%%%%%%%%%%%%%%%%%%%%%%%%%%%%%%%%%%%%%%%%%
\begin{abstract}
	More than one hundred years since the wireless communication was used for the first time, the wireless technology has impressively pervaded human life not only for communication but for the Internet use as well. For a long time, mobile devices use increase has burdened the traditional wire-based local network infrastructure due to its exorbitant escalation. Therefore, a number of companies, corporations and home users are deploying wireless system in local area network due to its mobility and scalability. However, wireless infrastructure raises many security and privacy concerns for business and scientists. As an attacker can catch the information gratuitously as long as he is in the signal range, authentication and data encryption is some of the most prioritized tasks in wireless network. In this report, we generally appraise several secure technologies in wireless local area network such as open shared system, Wired Equivalent Privacy, Temporal Key Integrity Protocol,... and some advanced technologies, which alleviates weaknesses of predecessors.
\end{abstract}
\subfile{intro.tex}
\subfile{wep}
\subfile{tkip}
\subfile{wpa2}
\subfile{future.tex}
\subfile{conclusion.tex}


\addtolength{\textheight}{-12cm}   % This command serves to balance the column lengths
                                  % on the last page of the document manually. It shortens
                                  % the textheight of the last page by a suitable amount.
                                  % This command does not take effect until the next page
                                  % so it should come on the page before the last. Make
                                  % sure that you do not shorten the textheight too much.

%%%%%%%%%%%%%%%%%%%%%%%%%%%%%%%%%%%%%%%%%%%%%%%%%%%%%%%%%%%%%%%%%%%%%%%%%%%%%%%%



%%%%%%%%%%%%%%%%%%%%%%%%%%%%%%%%%%%%%%%%%%%%%%%%%%%%%%%%%%%%%%%%%%%%%%%%%%%%%%%%



%%%%%%%%%%%%%%%%%%%%%%%%%%%%%%%%%%%%%%%%%%%%%%%%%%%%%%%%%%%%%%%%%%%%%%%%%%%%%%%%
\section*{APPENDIX}

All the references are available in this link: \url{https://drive.google.com/drive/folders/19h5EtdV93lfnSevDG2-MDS90Ph91fzLM?usp=sharing} 

%%%%%%%%%%%%%%%%%%%%%%%%%%%%%%%%%%%%%%%%%%%%%%%%%%%%%%%%%%%%%%%%%%%%%%%%%%%%%%%%

\printbibliography


\end{document}

