\section{Wired Equivalent Protocol - WEP}


\ac{WEP} is a security protocol, defined in \ac{IEEE}802.11, to secure link-level data throughout wireless transmission. Its message encryption and decryption mechanism was aimed to provide confidentiality, access control and data integrity from data-link layer.

\ac{WEP} mechanism is based on the \ac{RC4} algorithm\cite{mousa2006evaluation} that aims to supply security to \ac{WLAN} equivalent to \ac{LAN}. In general, the \ac{WEP} encryption of a frame includes three steps described in details in \ac{IEEE}802.11 standard \cite{al2006ieee}. The encryption algorithm takes input as a plaintext $M$ and produce the output $C$ and transmit over the network. The three steps of \ac{WEP} would be described as below
\begin{steps}
	\item \ac{WEP} calculate checksum $c(M)$ and concatenate with the original plaintext $M$, we have the message $P = <M, c(M)>$
	\item \ac{WEP} would encrypt the message $P$ achieved in the first step using \ac{RC4}. \ac{RC4} generates a keystream relying on the \ac{IV} and a key $k$. The keystream is denoted as $RC4(v,k)$. We use \ac{Xor}on the plain text and key stream to get the output $C$ of this step is $C = P \oplus RC4(v,k)$
	\item Concatenate $C$ achieved in step 2 with the \ac{IV} and transmit $<C, v>$ over the network.
\end{steps}

An \ac{AP} would catch the message and decrypt it by using \ac{Xor} operation over the encrypted message with \ac{RC4} key stream.  First, \ac{WEP} would extract message in to $v$ and $C$. After that, 