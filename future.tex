\section{Future Wireless Security Development}
\subsection{WPA3}
Wi-Fi Certified WPA3 is the next generation of Wi-Fi security for both personal and enterprise networks. WPA3 delivers a suite of features that simplify Wi-Fi security, provide more robust authentication, and deliver increased cryptographic strength. WPA3-Personal provides robust, password-based authentication for personal, such as home, networks. WPA3-Enterprise offers enterprise-grade security for sensitive data networks like those found in government, military, or finance. \cite{kinney_2018}
\subsection{Differences between WPA2 and WPA3}
The new WPA3 has surpassed WPA2 standard in many points, but below is 4 main points that have been summed up by the Wi-Fi Alliance:
\subsubsection{Protect from KRACK attack}~\\
Usually the current technology uses a passphrase to connect to the network via router. The process of giving and receiving password is like a handshake between the client devices and the network system. However, in the year of 2017, a researcher found out that people can exploit from the handshake process to enter the network without a passphrase.
The WPA3 provides a new handshake process, that ensure that even user enter a minimum short password, that password is ensured that can not be discovered by BruteForce attack. \cite{wiggers_2018}
\subsubsection{Easy connectivity}~\\
In recent years, with the rise of new technology, many devices are released to the market. Those devices includes those which are able to connect to WiFi and send and receive signals. However, people often feel disgusted when attempt to connect those devices with their WiFi because there often no display for those devices. In addition, manufacturers ask consumers to use a smartphone application to connect the device with the network.
WPA3 is believed that simplifies the process even the companies have not yet explained about this. \cite{kinney_2018}
\subsubsection{Public network privacy}~\\
Public networks are the ones that people might find at the public places that often do not ask user to provide a password to connect to. The hacker could make use of this simplification to secret transmit the user’s data that logged in the network.
WPA3 comes to solve this problem by encrypting the connection of the user to the network leaving no places for hacker attempting other’s data.\cite{hoffman_2018} 
\subsubsection{Security for governments}~\\
Not only the public security is benefitted from WPA3, but WPA3  is hoped that provide a new level of security for national organization with it’s far more complex encryption in comparison with old version of WPA.
\subsection{WPA3 is officially used }
There comes the question how long the WPA3 goes into use. Companies expect to deploy WPA3 in late 2018. Furthermore, manufacturers might have to choose between updating the old devices which currently support WPA2 or older versions or releasing totally new devices installed WPA3 \cite{hoffman_2018}.