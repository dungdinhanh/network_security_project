\section{INTRODUCTION}

\ac{LAN} \cite{stallings1984local} is a network system that connects electronic devices in a limited area or organizations such as school, building,...  LAN has prevailed societies for many years as it provides a means of fast communication or data transmitting \cite{schultheis1988designing}. It did contribute greatly to the economic growth in the end of $XX$ century and the begin of $XXI$ century.

However, traditional \ac{LAN} system remains many drawbacks. The wired technology causes the installation of \ac{LAN} not only to be exorbitant but also too complicated in huge system. In order to set up a \ac{LAN} system, an organization might have to concern about the design of the building, a tons of wires and the estimated number of gadgets. Those arduous tasks have  a detrimental effect on the tremendous increase in the installation cost of \ac{LAN}. Furthermore, the un-scalability of \ac{LAN} is another concern as there are more and more devices demanding to gain access to networks, especially mobile devices. A device, which wants to connect to \ac{LAN} may at least have to have a \ac{LAN} cable and an available port on the switch device. This would end up in the complication of system. Hence, \ac{LAN} seems to be not efficient.

\ac{WLAN} seems to cover most of the shortcomings of \ac{LAN} \cite{al2006ieee}.  As it is a wireless technology, \ac{WLAN} physical requirement is not too complicate. It is often associated with lower installation cost, so it is really beneficial to organizations' spending. Moreover, \ac{WLAN} is a feasible solution for escalation as it provides wireless network access at viable acceptable data rates \cite{al2006ieee}. People often deploy \ac{WLAN} as an extension to existing traditional \ac{LAN} which absolutely reduce the escalation cost compared to wired \ac{LAN} system expending.

\ac{WLAN}, on the other hand, faces a serious security issue due to its radio frequency. Messages are broad cast to everywhere inside the coverage area of \ac{WLAN} \cite{ieee1999802, shin2006wireless}. While \ac{LAN} is able to block and manage data transmission through cables to some extent. \ac{WLAN} might provide an access point to an adversary as airwaves propagation can not be clogged. Consequently, \ac{WLAN} faces a big risk of eavesdropping and Man-in-the-middle attacks \cite{shunman2003wlan}, so security protocols are of paramount importance. In order to bring the safety to \ac{WLAN}, the security protocols must satisfy three goals, authentication, confidentiality and integrity of data transmission \cite{gast2005802}. In this paper, we chiefly focus on the security protocols of \ac{WLAN}.

Organizations almost deployed \ac{WLAN} follow \ac{IEEE}802.11 standard \cite{ieee1999802}, yet it raised several security concerns. The flaws in designs in \ac{WEP} provided by \ac{IEEE}802.11 encourage any adversary to attack the system passively or actively. Furthermore, other difficult issues such as key-management and robust authentication really reduce the confidentiality of data transmission over network system \cite{arbaugh2002your}.

\ac{WPA} was introduced in 2003 \cite{fitzpatrick_2016} as the draft \ac{IEEE}802.11i standard by Wi-fi Alliance. Taking \ac{TKIP} as the core, \ac{WPA} ameliorates the security holes of \ac{WEP} deploying before. \ac{TKIP} is a set of algorithms to wrap \ac{WEP} to overcome some design constraints such as \ac{CPU} performance, \ac{WEP} hardwired encryption algorithm and dependence of software upgrade \cite{doomun2012modified}. Nevertheless, \ac{TKIP}, same as predecessor \ac{WEP}, is vulnerable to intrusion, so it makes \ac{WPA} weak and less confidential.As a result, \ac{WPA} is replaced by \ac{WPA}2 in 2006 \cite{fitzpatrick_2016}. Henceforth, \ac{WPA}2 has become endemic in every society from business environment to home environment as it is more resistant to network attack.

